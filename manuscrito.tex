\documentclass[11pt,]{article}
\usepackage[left=1in,top=1in,right=1in,bottom=1in]{geometry}
\newcommand*{\authorfont}{\fontfamily{phv}\selectfont}
\usepackage[]{mathpazo}


  \usepackage[T1]{fontenc}
  \usepackage[utf8]{inputenc}



\usepackage{abstract}
\renewcommand{\abstractname}{}    % clear the title
\renewcommand{\absnamepos}{empty} % originally center

\renewenvironment{abstract}
 {{%
    \setlength{\leftmargin}{0mm}
    \setlength{\rightmargin}{\leftmargin}%
  }%
  \relax}
 {\endlist}

\makeatletter
\def\@maketitle{%
  \newpage
%  \null
%  \vskip 2em%
%  \begin{center}%
  \let \footnote \thanks
    {\fontsize{18}{20}\selectfont\raggedright  \setlength{\parindent}{0pt} \@title \par}%
}
%\fi
\makeatother




\setcounter{secnumdepth}{3}


\usepackage{graphicx,grffile}
\makeatletter
\def\maxwidth{\ifdim\Gin@nat@width>\linewidth\linewidth\else\Gin@nat@width\fi}
\def\maxheight{\ifdim\Gin@nat@height>\textheight\textheight\else\Gin@nat@height\fi}
\makeatother
% Scale images if necessary, so that they will not overflow the page
% margins by default, and it is still possible to overwrite the defaults
% using explicit options in \includegraphics[width, height, ...]{}
\setkeys{Gin}{width=\maxwidth,height=\maxheight,keepaspectratio}

\title{Mi playa\\
Subtítulo\\
Subtítulo  }



\author{\Large Ana Hilda Valera Arias\vspace{0.05in} \newline\normalsize\emph{Estudiante, Universidad Autónoma de Santo Domingo (UASD)}  }


\date{}

\usepackage{titlesec}

\titleformat*{\section}{\normalsize\bfseries}
\titleformat*{\subsection}{\normalsize\itshape}
\titleformat*{\subsubsection}{\normalsize\itshape}
\titleformat*{\paragraph}{\normalsize\itshape}
\titleformat*{\subparagraph}{\normalsize\itshape}

\titlespacing{\section}
{0pt}{36pt}{0pt}
\titlespacing{\subsection}
{0pt}{36pt}{0pt}
\titlespacing{\subsubsection}
{0pt}{36pt}{0pt}





\newtheorem{hypothesis}{Hypothesis}
\usepackage{setspace}

\makeatletter
\@ifpackageloaded{hyperref}{}{%
\ifxetex
  \PassOptionsToPackage{hyphens}{url}\usepackage[setpagesize=false, % page size defined by xetex
              unicode=false, % unicode breaks when used with xetex
              xetex]{hyperref}
\else
  \PassOptionsToPackage{hyphens}{url}\usepackage[unicode=true]{hyperref}
\fi
}

\@ifpackageloaded{color}{
    \PassOptionsToPackage{usenames,dvipsnames}{color}
}{%
    \usepackage[usenames,dvipsnames]{color}
}
\makeatother
\hypersetup{breaklinks=true,
            bookmarks=true,
            pdfauthor={Ana Hilda Valera Arias (Estudiante, Universidad Autónoma de Santo Domingo (UASD))},
             pdfkeywords = {dinámica costera, \emph{beachrock}, manglar,erosión, playa},  
            pdftitle={Mi playa\\
Subtítulo\\
Subtítulo},
            colorlinks=true,
            citecolor=blue,
            urlcolor=blue,
            linkcolor=magenta,
            pdfborder={0 0 0}}
\urlstyle{same}  % don't use monospace font for urls

% set default figure placement to htbp
\makeatletter
\def\fps@figure{htbp}
\makeatother

\usepackage{pdflscape} \newcommand{\blandscape}{\begin{landscape}}
\newcommand{\elandscape}{\end{landscape}}


% add tightlist ----------
\providecommand{\tightlist}{%
\setlength{\itemsep}{0pt}\setlength{\parskip}{0pt}}

\begin{document}
	
% \pagenumbering{arabic}% resets `page` counter to 1 
%
% \maketitle

{% \usefont{T1}{pnc}{m}{n}
\setlength{\parindent}{0pt}
\thispagestyle{plain}
{\fontsize{18}{20}\selectfont\raggedright 
\maketitle  % title \par  

}

{
   \vskip 13.5pt\relax \normalsize\fontsize{11}{12} 
\textbf{\authorfont Ana Hilda Valera Arias} \hskip 15pt \emph{\small Estudiante, Universidad Autónoma de Santo Domingo (UASD)}   

}

}








\begin{abstract}

    \hbox{\vrule height .2pt width 39.14pc}

    \vskip 8.5pt % \small 

\noindent Mi resumen


\vskip 8.5pt \noindent \emph{Keywords}: dinámica costera, \emph{beachrock}, manglar,erosión, playa \par

    \hbox{\vrule height .2pt width 39.14pc}



\end{abstract}


\vskip 6.5pt


\noindent  \section{Introducción}\label{introducciuxf3n}

El mar constituye un elemento fundamental del conjunto de componentes de
la superficie terrestre, capaz de generar cambios en las líneas de
costas, sean estas en una isla o continente (referncia). De acuerdo con
Suárez de Vivero (1999), el termino costa se puede aludir a la franja de
tierra que bordea el mar o a la zona de contacto entre el medio marino y
el medio terrestre. Teniendo en cuenta que la línea de costa puede
variar en un instante, o con el paso de los años, ya sea por la dinámica
litoral o por causa de fenomenos naturales, que pueden traer como
posible concecuencia la erosión o regresión de la costa (Codignotto,
1997; Kokot, 2004).

Para Kokot (2004), la erosión costera es el resultado de un exceso de
remoción de sedimentos respecto del aporte suministrado a un área
determinada en un periodo específico. La misma abarca la emersión y
sumersión de sedimentos en las orillas del mar o la playa, lo que
mantiene en constante movimiento el límite exacto de la costa. Varios
autores se han dedicado al análisis de línea de costa, usando como
fuentes imágenes satelitales o fotografías áereas históricas. También se
realizan observaciones y mediciones por un periodo de tiempo determinado
que puedan dar respuesta a las causas de dicho cambio ({\textbf{???}};
Esquer, Carreon, \& others, 2018; Hernández Santana, Ortiz Pérez, Méndez
Linares, \& Gama Campillo, 2008).

La costa como unidad geomorfológica se mantiene en constante estado de
evolución. La importancia de conocer hacia dónde se desplaza más y qué
forma ésta va adquiriendo, permite diferenciar el tipo de costa que, de
acuerdo con Codignotto (1997), puede clasificarse como: costa en
progradación, costa estacionaria y costa en retrogradación. Del mismo
modo, el autor hace énfasis en la importancia de comprender los factores
que iniciden en este proceso y las causas que lo producen. Además de
incluir posible formación geoquimica que se puede producir en la zona
producto de estos cambios, como es el caso de la roca de playa.

De acuerdo con Aliotta, Spagnuolo, \& Farinati (2009), las rocas o
\emph{beachrock} son formaciones sedimentológicas comunes que evidencian
un proceso erosivo del litoral, los cuales se dieron lugar en un
ambiente geoquímico que enmarcó un periodo de evolución continuo que
pudo abarcar varias etapas del tiempo geológico. Dónde en tal proceso la
arena pudo ser compactada por medio de cemento carbonático y al pasar
varias épocas posiblemete afloraron. En la isla de Santo Domingo las
formaciones arrecifales o rocas de playas datan del Neógeno y el periodo
cuaternario. Ejemplo según ({\textbf{???}}), la Fm. Isabela del
pleistoceno; formación carbonatada arrecifal, rica en corales de tallas
variables. Aflora bajo la forma de diferentes relieves, formando
arrecifes en escalera descendiendo hacia el mar.

El litoral costero de la parte sur del país se caracteriza por pequeños
acantilados, playas de origen aluvial y dunas extensas (Abreu, 1999).
Además, mareas con oleajes extremos típico del mar caribe. No obstante,
la ecología actúa como componente categórico en el microclima de una
zona, resultado de la diversidad que ésta puede aportar. Por tal motivo,
el interés de conocer el tipo de vegetación. Razón de que estos, sobre
la arena son imprescindible para la conservación de los sedimentos, los
cuales pueden desvanecerse a concecuencia de la erosión del viento y la
lluvia (D'Croz, 1985).

De acuerdo con Cámara Artigas (1997), los litorales de la isla, se
caracterizan por tener plantas propias de la especie árboreas o
Rhizophoraceae como la morinda citrifolia (Noni), (ver figura
\ref{noni}) y el mangle rojo (ver figura \ref{manglerojo}). De igual
modo la vegetación cercanas a aguas dulce o salada suele llamarse
bosques de manglares, estos suelen encontrarse en algunas dunas costeras
de la parte sur del país, principalmente en las riveras y desembocaduras
de cuencas lacustre. Conforme Polanía \& Nat (1998), estos tipos de
bosques son asociaciones vegetales que prosperan en las costas
tropicales y subtropicales del mundo. Pero en la isla de Santo Domingo
existe una tipología diferente en dichos espacios costeros.

La playa de Najayo se encuentra ubicada en la sección del mismo nombre,
perteneciente al municipio San Gregorio de Nigua, provincia San
Cristóbal, al Sur de la República Dominicana. Fisiográficamente, se
ubica en la llanura costera del Caribe, en las coordenadas aproximadas
18º17'40" latitud Norte y 70º06'02" longitud Oeste. De acuerdo al mapa
geológico de la isla de Santo Domingo (Abad de los Santos, 2007--2010),
se estima que la formación del relieve costero de Najayo data de la era
Cenozoica periodo Cuaternario entre las época Eoceno-Mioceno, el mismo
está compuesto por arena y gravas bioclásticas formando el cordón
litoral, además de conglomerado, gravas, arenas de fondo de valle,
calizas arrecifales, calciruditas y calcarenitas (ver figura
\ref{mapageo50k}).

\begin{figure}
\centering
\includegraphics{noni.png}
\caption{Vegetación dunas de playa\label{noni}}
\end{figure}

\begin{figure}
\centering
\includegraphics{mangle_rojo.png}
\caption{Vegetación riveras de playa-río\label{manglerojo}}
\end{figure}

\begin{figure}
\centering
\includegraphics{mapa_bahia_najayo.png}
\caption{Mapa geológico escala 1:50,000 (hoja Nizao)\label{mapageo50k}}
\end{figure}

\ldots

\section{Metodología}\label{metodologuxeda}

Para el análisis de cambio en la línea costera de la playa Carlos Pinto,
ubicada en el paraje del mismo nombre en la sección Playa Najayo
provincia San Cristóbal. Se utilizó como referencia de estudio imágenes
satélitales de Landsat 5, 7 y 8, de los años
(2013,2014,2015,2016,2017,2018,2019). Las cuales fueron delimitadas
empleando el algoritmo de CoastSates que de acuerdo con Elsevier (n.d.)
es un conjunto de herramientas de software de código abierto escrito en
Python que permite al usuario obtener series de tiempo, pueden ser estos
de 30 o más años, en cuanto a la posición de una costa sin importar que
sea de tipo arenosa y a nivel mundial. Dicho software toma como base de
datos imágenes satelitales disponibles al público. También se colectaron
arenas y gravas en varios puntos de la costa, donde se llenó un
formulario y se tomó las coordenadas geograficas de cada punto por medio
de la aplicación ODK Collection descargada en un dispositivo móvil.
Tales puntos fueron identificado por área con respecto al mar o la playa
(Berma y Dunas de Playa), además de emplear los puntos cardinales para
tal ubicación. Los clastos colectados fueron medidos en dos ejes (ancho
y largo), de tal modo los resultados obtenidos fueron expresados en
milímetros (mm) como unidad de medida. De igual manera se fotografió
mediante la cámara de un teléfono móvil la vegetación y roca cercana a
la costa.

\ldots

\section{Resultados}\label{resultados}

\ldots

\section{Discusión}\label{discusiuxf3n}

\section{Agradecimientos}\label{agradecimientos}

\section{Información de soporte}\label{informaciuxf3n-de-soporte}

\ldots

\section{\texorpdfstring{\emph{Script}
reproducible}{Script reproducible}}\label{script-reproducible}

\ldots

\section*{Referencias}\label{referencias}
\addcontentsline{toc}{section}{Referencias}

\hypertarget{refs}{}
\hypertarget{ref-abad2007mapageonizao}{}
Abad de los Santos, M. (. (2007--2010). \emph{Mapa Geológico de la
República Dominicana a escala 1:50.000 de la hoja n 6170-I (Nizao) y
Memoria correspondiente}. Santo Domingo: Proyecto 1B de Cartografía
Geotemática de la República Dominicana. Programa SYSMIN. Servicio
Geológico Nacional.

\hypertarget{ref-abreu1999impacto}{}
Abreu, L. (1999). Impacto del turismo en el litoral de dominicana.
\emph{Revista Geográfica}, 167--182.

\hypertarget{ref-aliotta2009origen}{}
Aliotta, S., Spagnuolo, J. O., \& Farinati, E. A. (2009). Origen de una
roca de playa en la región costera de bahía blanca, argentina.
\emph{Pesquisas Em Geociências}, \emph{36}(1), 107--116.

\hypertarget{ref-camara1997republica}{}
Cámara Artigas, R. (1997). \emph{República dominicana: Dinámica del
medio físico en la región caribe (geografía física, sabanas y litoral)
aportación al conocimiento de la tropicalidad insular}.

\hypertarget{ref-codignotto1997geomorfologia}{}
Codignotto, J. (1997). \emph{Geomorfología y dinámica costera}.

\hypertarget{ref-d1985manglares}{}
D'Croz, L. (1985). Manglares: Su importancia para la zona costera
tropical. \emph{Agonia de La Naturaleza}, 167--180.

\hypertarget{ref-vos2019coastsat}{}
Elsevier (Ed.). (n.d.). CoastSat: Un kit de herramientas python
habilitado para google earth engine para extraer costas de imágenes
satelitales disponibles públicamente. \emph{Environmental Modeling ~\&
Software}.

\hypertarget{ref-esquer2018modificacion}{}
Esquer, M. Z., Carreon, T. E., \& others. (2018). MODIFICACION de linea
de costa. \emph{Revista de Investigación Académica Sin Frontera:
División de Ciencias Económicas Y Sociales}, (16).

\hypertarget{ref-hernandez2008morfodinamica}{}
Hernández Santana, J. R., Ortiz Pérez, M. A., Méndez Linares, A. P., \&
Gama Campillo, L. (2008). Morfodinámica de la línea de costa del estado
de tabasco, méxico: Tendencias desde la segunda mitad del siglo xx hasta
el presente. \emph{Investigaciones Geográficas}, (65), 7--21.

\hypertarget{ref-kokot2004erosion}{}
Kokot, R. R. (2004). \emph{Erosión en la costa patagónica por cambio
climático}.

\hypertarget{ref-polania1998manejo}{}
Polanía, J., \& Nat, R. (1998). Manejo de ecosistemas de manglar.
\emph{Memorias Del Curso Manejo de Ecosistemas de Manglar Y Arrecifes de
Coral. Bogotá}, 153--168.

\hypertarget{ref-suarez1999delimitacion}{}
Suárez de Vivero, J. L. (1999). Delimitación y definición del espacio
litoral. \emph{Jornadas Sobre El Litoral de Almería: Caracterización,
Ordenación Y Gestión de Un Espacio Geográfico Celebradas En Almería, 20
a 24 de Mayo de 1997. Pag: 13-23}.




\newpage
\singlespacing 
\end{document}
